\documentclass[conference]{IEEEtran}
\usepackage{graphicx} % Required for inserting images

\title{Smart City Assen}
\author{Ciaran Maher, Calvin Bootsman, Fotios Koutkos}

\begin{document}

\maketitle
\section{Introduction}
In ordor to meet its commitments to the Paris Accord, the Netherlands is rapidly transitioning to a renewable energy mix, requiring a fundamental redesign of its electricity, oil, and heating infrastructure\cite{paris_agreement}. This national transition necessitates significant changes at the local level, particularly in urban areas like Assen(population ~69,701 in 2024\cite{allecijfers}), where residential heating plays a crucial role in overall energy consumption. Detailed energy consumption data for Assen underscores the city's significant reliance on natural gas for residential heating, highlighting a crucial opportunity for carbon emission reduction. The detached houses in Assen consume an average of 1,490 m$^3$ of natural gas annually, while the apartments consume 670 m$^3$\cite{assen_woningen}, indicating a substantial variation in heating demands across different housing types. Converting these natural gas consumption figures to energy units reveals that approximately 80\% of the energy consumption is used for heating.

In urban areas like Assen, district heating presents a significant opportunity to transition towards sustainable energy consumption. Residential heating constitutes a major portion of the city's energy demand, particularly during colder months. By establishing a robust district heating network powered by renewable energy sources, Assen can significantly reduce its carbon footprint and enhance energy efficiency\cite{allecijfers}.

This proposal aims to develop a smart energy management system that implements Internet of Things (IoT) technologies to optimize a renewable-powered district heating network in Assen. The core of this system will focus on integrating diverse renewable energy sources, such as solar thermal, biomass, and heat pumps, to supply the district heating network. By integrating predictive weather analytics, the system will forecast heating demands and employ seasonal thermal energy storage to store surplus renewable heat generated during periods of high availability for use during peak demand.

While previous studies have demonstrated the feasibility of district heating in various European cities \cite{kalundborg}, the integration of IoT-based real-time monitoring and predictive controls, specifically tailored to maximize the use of renewable sources, remains an area of untapped potential. By establishing an intelligent, renewable-focused energy distribution network, this project seeks to enhance energy efficiency, reduce the dependence on fossil fuels, and contribute to the smart modernization of the city of Assen, ensuring a sustainable and resilient energy future.

\section{Literature Review}
\subsection{Heat sources for District Heating}
Waste heat recovery is a well-researched field with applications in district heating and sustainable energy solutions. Cities such as Kalundborg, Denmark, have successfully implemented industrial symbiosis networks where excess heat from power plants and industrial facilities is redirected to heat residential areas \cite{kalundborg}.

Waste heat recovery could be considered as a viable heat source for Assen's district heating network. however, based on the industries located within Assen\cite{bedrijvenopdekaart_assen}, this is not a viable option.

Instead, renewable energy sources such as solar thermal, biomass, and heat pumps could be utilized to power the district heating network\cite{nationaal_programma_duurzame}. Solar thermal systems can capture solar energy and convert it into heat, providing a sustainable and reliable energy source. Biomass boilers can burn organic materials to generate heat, offering a carbon-neutral alternative to fossil fuels. Heat pumps can extract heat from the air, ground, or water, providing an energy-efficient heating solution. By integrating these renewable sources, Assen can establish a resilient and sustainable district heating network that reduces carbon emissions and enhances energy efficiency.

\subsection{Electricity Auction Market in The Netherlands}
The Dutch electricity market employs blind auctions in day-ahead and intraday wholesale markets to determine hourly prices, balancing supply (diverse sources including renewables) with demand (households/businesses). TenneT, the transmission system operator, maintains real-time balance via balancing markets\cite{tennetmarketfacilitation}. While renewables compete in these auctions, the Sustainable Energy
Production Scheme and offshore wind auctions provide financial incentives\cite{dutchoffshorewind}, effectively prioritizing renewable energy development through subsidies and increasing market competitiveness. IoT is also used to monitor energy consumption on the household level, to help with the demand prediction\cite{cbs2022slimmeapparaten}, for example by using smart meters in houses which measures power consumption and solar power generation. 

\subsection{Climate Factors Influencing Energy Management}
The climate of Assen exhibits seasonal variations in temperature, precipitation, snowfall, and solar energy, all affecting energy demand and resource availability.

The warm season lasts 3.3 months, from June to September, with daily highs exceeding 19°C. July is the warmest month, averaging highs of 22°C and lows of 12°C. The cold season spans 3.7 months, from November to March, with daily highs below 8°C. January, the coldest month, records an average high of 5°C and a low of 0°C, highlighting winter heating needs. \cite{weatherspark_assen}

Precipitation influences moisture-related energy sources. A wet day, with at least 1 mm of precipitation, occurs more frequently from May to February, with December averaging 10.3 wet days. The drier season, from February to May, reaches a minimum in April at 7.0 wet days. Rainfall, the primary precipitation type, peaks at 34 percent in July, offering water-based energy recovery opportunities. \cite{weatherspark_assen}

Snowfall affects thermal storage and distribution. The snowy period lasts 1.7 months, from December to February, with accumulations exceeding 25 mm in any 31-day period. January sees the most snowfall, averaging 33 mm. The snowless period, spanning 10 months from February to December, limits snow-based thermal storage outside winter. \cite{weatherspark_assen}

Solar energy variations influence heating and renewable integration. The brighter period, from April to August, sees daily shortwave solar energy exceed 5.1 kWh/m², peaking in June at 6.3 kWh/m². The darker period, from October to February, drops below 1.7 kWh/m², reaching a December low of 0.5 kWh/m², emphasizing predictive models in heating management. \cite{weatherspark_assen}


\subsection{IoT in Smart Energy Management}
The integration of the Internet of Things (IoT) in energy management has led to the development of smart, data-driven energy systems. IoT sensors enable real-time monitoring of energy demand and optimize the distribution of heat based on consumption patterns and weather predictions\cite{smarthome_iot_big_data}. Studies have shown that AI-powered predictive analytics can enhance energy efficiency by adjusting heating systems in response to fluctuations in demand and external climate conditions \cite{ai_energy}. Furthermore, smart thermostats and automated controls have demonstrated significant energy savings in residential applications \cite{smart_heating}.

For an IoT-based district heating system, selecting an appropriate computational architecture is crucial to ensure efficient data processing and decision-making. IoT systems can be deployed using edge, fog, or cloud computing models\cite{s21175922}, each offering unique advantages depending on latency requirements, scalability, and power consumption. Edge computing enables real-time data processing at the sensor level, reducing reliance on external networks and minimizing latency, which is critical for heat distribution adjustments based on immediate weather and energy storage data. Fog computing provides a distributed processing approach, where intermediate nodes (such as local servers in district heating facilities) handle some computations while offloading complex processes to the cloud. In contrast, cloud computing centralizes all data processing, providing robust analytical capabilities but introducing potential latency and network dependency issues \cite{iot_architecture}. 

Communication technologies play an equally vital role in ensuring seamless data transmission between sensors, energy storage units, and control systems. In conventional heating networks, wired communication protocols such as Ethernet and Modbus are commonly used in industrial settings for reliable, high-speed data exchange. However, IoT-based smart city applications often require wireless solutions for greater flexibility and scalability. Technologies such as LoRaWAN and Wi-Fi are particularly relevant, each with trade-offs in coverage, power efficiency, and data transfer speed. LoRaWAN (Long Range Wide Area Network) is well-suited for large-scale use of low-power, long-range sensors in urban heating applications. Meanwhile, Wi-Fi is commonly used in smart home applications to control heating systems locally but has limitations in power efficiency and range \cite{comm_tech_review}. 

\section{Design}

The proposed smart heating system for Assen suggests an IoT-based predictive energy management framework to optimize the integration of renewable heat sources into a district heating network. This system continuously monitors environmental conditions, energy availability, and thermal storage levels to determine the most effective energy distribution strategy. A hybrid Edge-Fog architecture is selected to balance real-time decision-making at the local level with long-term forecasting in the cloud.

Real-time monitoring relies on a network of sensors deployed throughout the city. Weather sensors measure atmospheric parameters such as temperature, humidity, wind speed, and solar intensity to predict heating demand fluctuations. Heat source sensors track the availability of renewable energy from solar thermal, biomass, and heat pumps, ensuring efficient allocation of heating resources. Thermal storage sensors monitor temperature and capacity levels in underground reservoirs, enabling strategic retention and release of stored heat. These integrated components provide the data needed for dynamic heat management and efficient system operation.

\subsection{IoT Smart Heating and Forecasting}

To enhance heating system performance, the IoT framework incorporates machine learning algorithms trained on historical weather patterns and real-time sensor data. These predictive models enable accurate forecasting of heating demands, allowing the system to anticipate periods of surplus and scarcity. By analyzing climate trends, such as seasonal temperature variations and solar radiation intensity, the system can optimize energy storage and release cycles.

Advanced IoT-based control mechanisms dynamically adjust heat distribution in response to demand fluctuations. These automated controls ensure that excess energy is either stored or redirected to areas with higher demand, reducing unnecessary energy waste. By integrating forecasting with automated heat management, the system significantly improves efficiency and reduces reliance on backup fossil fuel sources.


\subsection{Heat Source Selection for District Heating}

The selection of heat sources plays a fundamental role in the performance of Assen’s district heating network. This proposal integrates solar thermal, biomass, and heat pumps as primary energy sources while utilizing natural gas as a transitional backup to ensure supply stability.

\textbf{Solar thermal systems} capture and convert solar radiation into heat, providing a sustainable energy source, particularly during the brighter months.
\textbf{Biomass Boilers} utilize organic materials to generate heat, offering a carbon-neutral alternative to fossil fuels and supporting national renewable energy goals.
\textbf{Heat pumps} extract heat from ambient air, ground, or water sources, delivering an energy-efficient heating solution suited to Assen’s climate.
\textbf{Natural gas} functions as a backup during peak demand periods, ensuring reliability while renewable capacity is scaled up.
By strategically integrating these sources, the district heating network minimizes carbon emissions while maintaining a stable and adaptable energy supply.

Advanced IoT controls adjust heat distribution based on demand forecasts, optimizing energy use and reducing waste. We propose a localized energy auction market, using IoT data to set hourly heat prices, mirroring the Dutch electricity market. Targeted subsidies will incentivize renewable heat adoption, balancing economic efficiency and sustainability in Assen's transition to carbon neutrality.

\subsection{System Integration and Communication Infrastructure}

Effective system operation relies on real-time data exchange between sensors, storage units, and control mechanisms. A robust communication infrastructure is required to transmit data efficiently while balancing energy consumption and network coverage.

LoRaWAN (Long Range Wide Area Network): Low-power, long-range communication protocol used for city-wide sensor deployments, enabling real-time environmental and heat source monitoring.
Wi-Fi: High-bandwidth, short-range protocol implemented in district heating facilities and residential units to facilitate localized energy control.
A hybrid communication strategy is adopted, utilizing LoRaWAN for large-scale sensor networks and Wi-Fi for real-time adjustments in heating distribution. This ensures efficient data flow across the network, supporting automated decision-making processes.

\subsection{Comparison of IoT Architectures and Communication Methods}

To evaluate the efficiency of different architectures and communication technologies, the following criteria were analyzed:

\begin{table}[h]
\centering
\caption{Comparison of IoT Architectures}
\begin{tabular}{|l|c|c|c|}
\hline
\textbf{Criteria} & \textbf{Edge} & \textbf{Fog} & \textbf{Cloud} \\
\hline
Latency & Low & Moderate & High \\
Scalability & Moderate & High & Very High \\
Power Consumption & High & Moderate & Low \\
Data Processing & Local & Distributed & Centralized \\
\hline
\end{tabular}
\end{table}

The Edge-Fog hybrid model is selected as it enables low-latency processing for real-time heat distribution while leveraging fog computing for regional data aggregation and optimization.

\begin{table}[h]
\centering
\caption{Comparison of Communication Technologies}
\begin{tabular}{|l|c|c|}
\hline
\textbf{Criteria} & \textbf{LoRaWAN} & \textbf{Wi-Fi} \\
\hline
Range & Long & Short \\
Power Consumption & Low & High \\
Data Rate & Low & High \\
Urban Suitability & High & Moderate \\
\hline
\end{tabular}
\end{table}

By integrating both communication protocols, the system maintains high efficiency and adaptability across varying operational scenarios.

\subsection{Implementation Challenges and System Optimization}

Several challenges must be addressed to ensure efficient deployment and long-term performance:

Sensor Calibration: Weather and heat sensors require periodic recalibration to maintain accuracy. Automated self-checking algorithms can detect and correct faults.
Thermal Storage Losses: Heat reservoirs experience gradual energy loss. Advanced insulation materials can mitigate this issue and improve heat retention.
Infrastructure Integration: Retrofitting existing buildings with IoT-enabled heating controls may require policy incentives and additional funding.
Addressing these challenges through system refinements and technological advancements enhances the feasibility and resilience of Assen’s district heating network.

\subsection{Conclusion and Future Work}

This paper presents an IoT-enabled predictive heating system for Assen, integrating real-time environmental monitoring, advanced forecasting, and adaptive energy distribution. The system’s Edge-Fog hybrid architecture enables efficient decision-making, while a LoRaWAN-Wi-Fi communication framework ensures reliable data exchange.

Future research could explore enhanced AI models for heat demand forecasting to further optimize energy use. Additionally, an innovative extension could involve utilizing excess heat from the district heating system to warm bike lanes in winter, reducing ice accumulation and improving safety. Further exploration of alternative thermal storage solutions, such as phase-change materials, could also enhance long-term energy efficiency.

\bibliographystyle{ieeetr}
\bibliography{bibliography}

\end{document}